\documentclass[a4paper]{article}

\usepackage[top=1.2in, bottom=1.2in, left=1in, right=1in]{geometry}
\usepackage{graphicx, minted}
\usepackage[T1]{fontenc}

\usemintedstyle{tango}

\title{LC2K Assembler}
\author{Rijun Cai\\12348003}

\begin{document}
\maketitle

\section{Introduction}
In this laboratory, an LC2K assembler was implemented in C++. In the following, I will give some detailed explaination
of my source code of the assembler.

\section{Source Code Explaination}
\subsection{Types \& Classes Summary}
To organized the program better, several types and classes are defined. Here is a list of them.

\begin{description}
    \item[\texttt{mc\_t}] An alias of \verb|uint_least32_t|. The type of a machine code.
    \item[\texttt{mc\_t\_s}] An alias of \verb|int_least32_t|. A signed version of \verb|mc_t|. Used to output.
    \item[\texttt{SyntaxError}] Exception to indicate the syntax errors in the input.
    \item[\texttt{IOError}] Exception to indicate the errors occur in IO\@.
    \item[\texttt{Assembler}] The class of an assembler. For more details, see~\ref{assembler_class}.
    \item[\texttt{Assembler.Instruction}] The class of an instruction.
\end{description}

\subsection{Workflow}
The \verb|main| function will first read the \verb|.asm| file with \verb|readFromFile| function.
Then an assembler will be created and fed with the strings of the input file. Then the assembler will analyse
the input strings with \verb|Assembler.synAnalyse|. Then the Pass One and Pass Two processes will be executed and
the machine codes will be generated. Finally, \verb|writeToFile| will be called to write the machine codes to the
output file. Here is the main part of the \verb|main| function.

\begin{minted}{cpp}
try
{
    vector<string> as = readFromFile(argv[1]);
    Assembler assembler;
    assembler.setAssembly(as);

    assembler.synAnalyse();
    assembler.passOne();
    assembler.passTwo();

    writeToFile(argv[2], assembler.getMechineCode());
}
catch(SyntaxError e)
{
    std::cerr << "Error occured when assembling.\n" << e.what() << std::endl;
    return EXIT_FAILURE;
}
catch(IOError e)
{
    std::cerr << "IOError occured: " << e.what() << std::endl;
    return EXIT_FAILURE;
}
\end{minted}

\subsection{\texttt{Assembler} Class in Detail}\label{assembler_class}
\verb|Assembler| class (along with \verb|Assembler.Instruction| structure) implements the main functions of the
assembler. Here are some details of it.

The \verb|*_INS| sets will be initialized when the assembler is constructed, and will be used to identify
the type of the instructions.

In the syntax analysis process (\verb|synAnalyse| function), every line of the assembly code will be
parsed as list of string tokens by \verb|std::stringstream|. Then the first token will be checked and
if it is a name of an instruction, the following fields will be treated as the fields of the instruction,
otherwise, the first token will be treated as a label. The result of \verb|synAnalyse| will be a list of
\verb|Instruction| (\verb|Assembler._ins|).

In \verb|passOne|, every label will be checked if it is duplicated and then mapped to the address of the
instruction. The result will be stored in \verb|Assembler._label_table|.

In \verb|passTwo|, every instruction will be passed to the corresponding translation function
and translated into a machine code.

The definitation of \verb|Assembler| class is as follow.
\begin{minted}{cpp}
class Assembler
{
    struct Instruction
    {
        Instruction() {}
        Instruction(
                const string &_label,
                const string &_instruction,
                const vector<string> _fields):
            label(_label), instruction(_instruction), fields(_fields) {}

        string label;
        string instruction;
        vector<string> fields;
    };

    private:
        vector<string> _as;
        vector<Instruction> _ins;
        map<string, size_t> _label_table;
        vector<mc_t> _mc;

        set<string> _INS;
        set<string> _R_INS;
        set<string> _I_INS;
        set<string> _J_INS;
        set<string> _O_INS;
        set<string> _PSEUDO_INS;

        inline int interpreOffsetField(const string &field,
                bool *pLabelSign = 0, bool half_word = true);

        inline mc_t translateAdd(const Instruction &ins);
        inline mc_t translateNand(const Instruction &ins);
        inline mc_t translateLw(const Instruction &ins);
        inline mc_t translateSw(const Instruction &ins);
        inline mc_t translateBeq(const Instruction &ins, size_t pc);
        inline mc_t translateJalr(const Instruction &ins);
        inline mc_t translateHalt(const Instruction &ins);
        inline mc_t translateNoop(const Instruction &ins);
        inline mc_t translateDotFill(const Instruction &ins);

        inline static mc_t convertRegisterField(const string &field);

    public:
        Assembler();
        void setAssembly(const vector<string> &as);
        void synAnalyse();
        void passOne();
        void passTwo();
        vector<mc_t> getMechineCode();
};
\end{minted}

\end{document}
