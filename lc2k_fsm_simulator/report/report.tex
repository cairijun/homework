\documentclass[a4paper]{article}

\usepackage[top=1.2in, bottom=1.2in, left=1in, right=1in]{geometry}
\usepackage{graphicx, minted}
\usepackage[T1]{fontenc}

\usemintedstyle{tango}

\title{LC2K FSM Simulator}
\author{Rijun Cai\\12348003}

\begin{document}
\maketitle

In this laboratory, an LC2K FSM simulator is implemented in C. It is based on the framework provided
in the lecture. I finished the FSM part of the simulator. This document introduces some details of
my implementation.

To keep the code neat and easy to read, some marcos are defined.

\begin{itemize}
    \item \verb|_S|

        \begin{minted}{text}
#define _S(name) name: printState(&state, #name)
        \end{minted}
        Begin to define a state. With this marco, a state can be simply defined as follow.
        \begin{minted}{cpp}
_S(fetch);
    bus = state.pc++;
    // The rest of fetch state
        \end{minted}
        The above code will be expanded as follow.
        \begin{minted}{cpp}
fetch: printState(&state, "fetch");
    bus = state.pc++;
    // The rest of fetch state
        \end{minted}


    \item \verb|_DISPATCH|

        \begin{minted}{text}
#define _DISPATCH(code, label) if((state.instrReg & 0x1C00000) == (code << 22)) goto label
        \end{minted}

        Define a branch of the dispatching process. Used in \verb|dispatch| state. With this marco,
        \verb|dispatch| state can be defined as follow.
        \begin{minted}{cpp}
_S(dispatch);
    _DISPATCH(0x0, add);
    _DISPATCH(0x1, nand);
    _DISPATCH(0x2, lw);
    _DISPATCH(0x3, sw);
    _DISPATCH(0x4, beq);
    _DISPATCH(0x5, jalr);
    _DISPATCH(0x6, halt);
    _DISPATCH(0x7, noop);
        \end{minted}
        The above code will be expanded as follow.
        \begin{minted}{cpp}
dispatch: printState(&state, "dispatch");
    if((state.instrReg & 0x1C00000) == (0x0 << 22)) goto add;
    if((state.instrReg & 0x1C00000) == (0x1 << 22)) goto nand;
    if((state.instrReg & 0x1C00000) == (0x2 << 22)) goto lw;
    if((state.instrReg & 0x1C00000) == (0x3 << 22)) goto sw;
    if((state.instrReg & 0x1C00000) == (0x4 << 22)) goto beq;
    if((state.instrReg & 0x1C00000) == (0x5 << 22)) goto jalr;
    if((state.instrReg & 0x1C00000) == (0x6 << 22)) goto halt;
    if((state.instrReg & 0x1C00000) == (0x7 << 22)) goto noop;
        \end{minted}


    \item \verb|_REG1|, \verb|_REG2|, \verb|_DES_REG| and \verb|_OFFSET|

        \begin{minted}{text}
#define _REG1 state.reg[(state.instrReg >> 19) & 0x7]
#define _REG2 state.reg[(state.instrReg >> 16) & 0x7]
#define _DES_REG state.reg[state.instrReg & 0x7]
#define _OFFSET convertNum(state.instrReg & 0xffff)
        \end{minted}
        Convenient access to registers and the offset field. Then reading and writing registers can
        be simplified as follow.
        \begin{minted}{cpp}
bus = _REG1;
_DES_REG = bus;
bus = _OFFSET;
        \end{minted}
        The above code will be expanded as below.
        \begin{minted}{cpp}
bus = state.reg[(state.instrReg >> 19) & 0x7];
state.reg[state.instrReg & 0x7] = bus;
bus = convertNum(state.instrReg & 0xffff);
        \end{minted}
\end{itemize}

\end{document}
